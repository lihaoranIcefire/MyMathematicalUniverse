\documentclass[main]{subfiles}

\begin{document}

Consider a system of $N$ particles in $\mathbb R^n$ with masses $m_i$, with positions $\mathbf r_i=\mathbf r_i(t)$, $\dot{\mathbf r_i},\ddot{\mathbf r_i}$ denote velocities and accelerations. Let $\mathbf x,t$ denote Cartesian coordinates and time, $\mathbf r_i$ may have some constraints(such as $f_j(\mathbf r_i,t)=0$), consider another coordinates(generalized coordinates, preferably free, i.e. with no constraints) $\mathbf y$, particles will have generalized positions, generalized velocities and generalized accelerations $\mathbf q_i=\mathbf q_i(t)$, $\dot{\mathbf q_i},\ddot{\mathbf q_i}$. The system itself should produce no work, i.e.
\begin{align*}
0&=\sum m_i\ddot{\mathbf r_i}\cdot\dot{\mathbf r_i} \\
&=\sum m_i\ddot{\mathbf q_i}\cdot\left[\frac{\partial\mathbf x}{\partial\mathbf y}(\mathbf q_i)\dot{\mathbf q_i}\right] \\
\end{align*}

\begin{example}

\end{example}

\begin{definition}

\end{definition}

\end{document}