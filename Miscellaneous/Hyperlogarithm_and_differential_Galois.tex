\documentclass[main]{subfiles}

\begin{document}

$X=\{x_0,\cdots, x_N\}$, $N\geq1$, $X^*$ is the free non-commutative monoid generated by $X$, $\mathbb C\langle\langle X\rangle\rangle$ is the ring of formal power series with non-commutative variables in $X$. Consider $j:X\to\mathbb C$, $\sigma_i=j(x_i)$, $\Sigma=j(X)\cup\{\infty\}$, $D=\mathbb{CP}^1-\Sigma$
The differential equation
\begin{equation}\label{10/31/2020-eq1}
\frac{\partial}{\partial z}F(z)=\sum_{i=0}^N\frac{x_i}{z-\sigma_i}F(z)
\end{equation}
is of Fuchs type. Where $F\in\mathcal O(D)\langle\langle X\rangle\rangle$, write $F(z)=\sum_{w\in X^*}F_w(z)w$, $F_w(z)\in\mathcal O(D)$, then the differential equation becomes
\begin{equation}\label{10/31/2020-eq2}
\frac{\partial}{\partial z}F_{x_kw}(z)=\frac{F_w(z)}{z-\sigma_k}
\end{equation}
Find half line $\ell(\sigma_i)$ starting at $\sigma_i$ such that they don't intersect, and $U=D-\bigcup\ell(\sigma_i)$ is a simply connected, pick a branch for $\log(z-\sigma_0)$ on $D-\ell(\sigma_0)$. $\hat D\xrightarrow p D$ is a the universal cover, when there is no confusion, we shall still use $L_w(z)$

\begin{theorem}
Equation \eqref{10/31/2020-eq1} has a unique solution
\[L(z)=f_0(z)e^{x_0\log(z-\sigma_0)}\]
such that $f_0$ is holomorphic on $D-\bigcup_{k\neq0}\ell(\sigma_k)$ with $f_0(\sigma_0)=1$ (We also write $L(z)\sim (z-\sigma_0)^{x_0}$ as $z\to\sigma_0$). And any other holomorphic solution can be uniquely written as $L(z)C$, where $C\in\mathbb C\langle\langle X\rangle\rangle$
\end{theorem}

\begin{proof}
For $w\in X^*$ not ending in $x_0$, write $w=x_0^{n_r-1}x_{i_r}\cdots x_0^{n_1-1}x_{i_1}$, define
\[L_w(z)=(-1)^r\sum_{1\leq m_1<\cdots<m_r}\frac{1}{m_1^{n_{i_1}}\cdots m_r^{n_{i_r}}}\left(\frac{z-\sigma_0}{\sigma_{i_1}-\sigma_0}\right)^{m_1}\cdots\left(\frac{z-\sigma_0}{\sigma_{i_r}-\sigma_0}\right)^{m_r-m_{r-1}}\]
in a neighborhood of $\sigma_0$. And analytically continue this by
\[L_{x_kw}(z)=\int_{\sigma_0}^z\frac{L_w(t)}{t-\sigma_k}dt\]
In particular
\[L_{x_i^n}(z)=\frac{1}{n!}\log^{n}\left(\frac{z-\sigma_i}{\sigma_0-\sigma_i}\right)\]

\[L_{x_0^n}(z)=\frac{1}{n!}\log^{n}\left(z-\sigma_0\right)\]
The $f_0(z)$ can be computed explicitly by comparing the coefficients in the series expansion that satisfies equation \eqref{10/31/2020-eq1}. $L(z)$ is invertible since its leading coefficeint is 1, suppose $F(Z)$ is another solution, define $K(z)=L(z)^{-1}F(z)$, then differentiate $F(z)=L(z)K(z)$ we have
\[F(z)'=L(z)'K(z)+L(z)K(z)'=\sum_{i=0}^N\frac{x_i}{z-\sigma_i}L(z)K(z)+L(z)K(z)'=F(z)'+L(z)K(z)'\]
Thus $L(z)K(z)'=0\Rightarrow K(z)'=0\Rightarrow K(z)=C$
\end{proof}

\begin{remark}
Heuristically, the solution might seem to be
\[\exp\left({x_0\log(z-\sigma_0)+\sum_{i=1}^Nx_i\log\frac{z-\sigma_i}{\sigma_0-\sigma_i}}\right)\]
But this violates the non-commutativity of variables $x_i$
\end{remark}

\begin{example}
In the case $\sigma_0=0,\sigma_1=1$, $L_{x_0^m}(z)=\frac{1}{n!}\log^nz$ and $L_{x_1^{n-1}x_0}(z)=-\Li_n(z)$, $L_{x_0^{n_1-1}x_{1}\cdots x_0^{n_d-1}x_{1}}(z)=(-1)^{n_1+\cdots+n_d-d}\Li_{n_1,\cdots,n_d}(z)$
\end{example}

\begin{corollary}
By translation, there is a unique solution to equation \eqref{10/31/2020-eq1}
\[L^{\sigma_k}(z)=f_k(z)e^{x_k\log(z-\sigma_k)}\]
over $\mathbb C-\bigcup_{i\neq k}\ell(\sigma_i)$ with $f_k(\sigma_k)=1$
\end{corollary}

\begin{proof}
Consider $z'-\sigma_0=e^{i\theta}(z-\sigma_k)$, $L^{\sigma_k}(z)=L(z')$, then
\[\frac{\partial}{\partial z}L^{\sigma_k}(z)=\sum_{i=0}^N\frac{x_i}{z-\sigma_i}L^{\sigma_k}(z)\]
Becomes
\begin{align*}
\frac{\partial}{\partial z'}L(z')&=\sum_{i\neq k}\frac{x_i}{z'-(\sigma_i-e^{i\theta}(\sigma_k-\sigma_i))}L(z')+\frac{x_k}{z'-\sigma_0}L(z')
\end{align*}
Hence
\begin{align*}
L^{\sigma_k}(z)=L(z')=f_0(z')e^{x_k\log(z'-\sigma_0)}=f_k(z)e^{x_k\log(z-\sigma_k)}
\end{align*}
\end{proof}

\begin{corollary}

\end{corollary}

\begin{proof}
Consider $z'-\sigma_0=z^{-1}$, $L^{\infty}(z)=L(z')$, then
\[\frac{\partial}{\partial z}L^{\infty}(z)=\sum_{i=0}^N\frac{x_i}{z-\sigma_i}L^{\infty}(z)\]
Becomes
\begin{align*}
\frac{\partial}{\partial z'}L(z')&=-\sum_{i=0}^\infty\frac{x_i}{(z'-\sigma_0)-(z'-\sigma_0)^2\sigma_i}L(z') \\
&=\frac{-x_\infty}{z-\sigma_0}L(z')+\sum_{i=0}^\infty\frac{x_i}{z-(\sigma_0+\sigma_i^{-1})}L(z')
\end{align*}
Here $x_\infty=\sum_{i=0}^N x_i$. Hence
\begin{align*}
L^{\sigma_k}(z)=L(z')=f_0(z')e^{x_k\log(z'-\sigma_0)}=f_\infty(z)e^{x_\infty\log z}
\end{align*}
\end{proof}

$\mathbb C\langle X\rangle$ is the free $\mathbb C$ algebra generated by non commuting variables of $X$, there is a shuffle product $\shuffle:\mathbb C\langle X\rangle\otimes\mathbb C\langle X\rangle\to\mathbb C\langle X\rangle$ defined inductively by
\[e\shuffle w=w\shuffle e=w\]
\[xw\shuffle yv=x(w\shuffle yv)+y(xw\shuffle v)\]
$\mathbb C\langle X\rangle\to\mathcal O(D)$, $w\mapsto L_w(z)$ is then a homomorphism, i.e. $L_{w\shuffle v}(z)=L_w(z)L_v(z)$. Let
\[\mathcal O_\Sigma=\mathbb C\left[z,\frac{1}{z-\sigma_0},\cdots,\frac{1}{z-\sigma_N}\right]\]
be the ring of regular functions on $D$. Let $L_\Sigma$ be the free $\mathcal O_\Sigma$ module generated by $L_w(z)$. $L_\Sigma$ is multiplicatively closed due to the shuffle relation, it is a differential algebra with derivation $\frac{\partial}{\partial z}$, and it is independent of the choice of the choice of branch of $\log$

For $0\leq k\leq N$, define truncate operator $\partial_k:\mathbb C\langle X\rangle\to\mathbb C\langle X\rangle$, $\partial_k(x_jw)=\delta_{jk}w$, it is easy to show that $\partial_k$ is a derivation for shuffle product by induction
\begin{align*}
\partial_k(x_iw\shuffle x_jv)&=\partial_k(x_i(w\shuffle x_jv)+x_j(x_iw\shuffle v)) \\
&=\delta_{ik}w\shuffle x_jv+\delta_{jk}x_iw\shuffle v \\
&=\partial_k(x_iw)\shuffle x_jv+x_iw\shuffle\partial_k(x_jv)
\end{align*}
The universal algebra of hyperlogarithm is the differential $\mathcal O_\Sigma$ algebra
\[\mathcal{HL}_\Sigma=\mathcal O_\Sigma\otimes_{\mathbb C}\mathbb C\langle X\rangle\]
With the induced shuffle product as multiplication and $\partial=\frac{\partial}{\partial z}\otimes1+\sum_{k=0}^N\frac{1}{z-\sigma_k}\otimes\partial_k$ as derivation

It is well known that shuffle product is associative and commutative. Define Lyndon words $\mathbf{Lyn}(X)\subseteq X^*$ by assuming lexicographical order $x_0<\cdots<x_N$ and
\[w=x_{i_1}\cdots x_{i_n}\in\mathbf{Lyn}(X)\iff x_{i_1}\cdots x_{i_r}\leq x_{i_{r+1}}\cdots x_{i_n},\forall r\]

\begin{theorem}[Radford]\label{10/31/2020-Radford's theorem}
$\mathbb C\langle X\rangle$ with shuffle product is the free polynomial algebra over $\mathbf{Lyn}(X)$
\end{theorem}

For $w\in X^*$, define weight $|w|$ to be the number of symbols in $w$. The induced filtration of $\mathcal{HL}_\Sigma$ is
\[\mathcal{HL}_\Sigma^{(m)}=\left\{\sum_{|w|\leq m}f_w(z)\otimes w\middle|f_w(z)\in\mathcal O_\Sigma\right\}\]
Since $\partial\mathcal{HL}_\Sigma^{(m)}\subseteq \mathcal{HL}_\Sigma^{(m)}$, we can define $\gr_m\mathcal{HL}_\Sigma$, by Theorem \ref{10/31/2020-Radford's theorem}, $\gr_m\mathcal{HL}_\Sigma$ is generated by Lyndon words of weight $m$ which are polynomials in $N$ variables of degree $m$. For example, if $m=1$, the Lyndon words are $x_i$'s, $0\leq i\leq N$, if $m=2$, the Lyndon words are $x_ix_j$, $0\leq i<j\leq N$

$\mathbb C\langle X\rangle$ has two natural Hopf algebra structure. The first one has non-commutative multiplication given by concatenation, cocommutative coproduct uniquely given by that $x_i$ are primitive, i.e. $\Delta(x_i)=x_i\otimes1+1\otimes x_i$, counit $\epsilon$ given the projection onto the coefficient of the trivial word $e$ and antipode a given by $w\mapsto (-1)^{|w|}\tilde w$, $\tilde w$ is the reversal of $w$. The second one has commutative multiplication given by the shuffle product, coproduct given by $\displaystyle\Gamma(w)=\sum_{uv=w}u\otimes v$, same counit $\epsilon$ and antipode $w\mapsto(-1)^{|w|}w$. These two Hopf algebra structures are dual to each other

\begin{corollary}
The map $S:\mathbb C\langle X\rangle\to\mathbb C^\times$ is a homomorphism iff $S=\sum_{w\in X^*}S(w)w\in\mathbb C\langle\langle X\rangle\rangle$ is group like, i.e. $\Delta S=S\otimes S$
\end{corollary}

\begin{remark}
$S$ may not be group like, but there exists $S^\times$ such that $\Delta S^\times=S^\times\otimes S^\times$ and $S^\times(w)=S(w)$ for all $w\in\mathbf{Lyn}(X)$
\end{remark}

\begin{recall}
If $(A,\partial)$ is a differential ring, $A$ is differentially simple if $A$ is a simple module over its ring of differential operators $A[\partial]$, or equivalently that $A$ has no nontrivial differential ideal. A differential ideal $I\subseteq A$ is such that $\partial I\subseteq I$. The ring of constants in $A$ is $\ker\partial$. An equivalent condition for differential simplicity is for any $\theta\in A$, there exists $D_\theta\in A[\partial]$ such that $D_\theta\theta=1$, otherwise $A[\partial]\theta$ would be such a nontrivial ideal
\end{recall}

\begin{theorem}
Every differential $\mathcal O_\Sigma$ subalgebra of $\mathcal{HL}_\Sigma$ is differentially simple. The ring of constants of $\mathcal{HL}_\Sigma$ is $\mathbb C$ and each element in $\mathcal{HL}_\Sigma$ has a primitive, i.e. the following sequence is exact
\[0\to\mathbb C\to\mathcal{HL}_\Sigma\xrightarrow{\partial}\mathcal{HL}_\Sigma\to0\]
\end{theorem}

\begin{proof}

\end{proof}

\begin{definition}
The differential Galois group of extension $\mathcal{HL}_\Sigma/\mathcal O_\Sigma$ is $\Gal(\mathcal{HL}_\Sigma/\mathcal O_\Sigma)=\Aut_{\mathcal O_\Sigma[\partial]}\mathcal {HL}_\Sigma$
\end{definition}

\begin{note}
The differential Galois group is normally about the field of fractions of a differential ring, however this is fine here since $\Gal(\mathcal{HL}_\Sigma/\mathcal O_\Sigma)\cong\Gal(\Frac\mathcal{HL}_\Sigma/\mathbb C(z))$
\end{note}

\begin{theorem}
There is a canonical isomorphism of groups $\Gal(\mathcal{HL}_\Sigma/\mathcal O_\Sigma)\cong\{S\in\mathbb C\langle\langle X\rangle\rangle|\Delta S=S\otimes S\}$, the right hand side has the group low of concatenation. There are also natural bijections between both groups and $\Hom(\mathbf{Lyn}(X),\mathbb C)$
\end{theorem}

\begin{proof}

\end{proof}

Both groups have natural filtrations induced from the filtration by weight. Let $U_n$ be the group of endomorphisms of $\mathcal{HL}_\Sigma^{(n)}$ that commute with $\partial$ and $\shuffle$, then $U_n\cong\{S\in\mathbb C\langle\langle X\rangle\rangle/X^{n+1}\mathbb C\langle\langle X\rangle\rangle|\Delta S=S\otimes S\}$. Since any group like $S$ satisfies $(S-1)^{n+1}\in X^{n+1}\mathbb C\langle\langle X\rangle\rangle$, $U_n$ is a unipotent matrix group and
\[\Gal(\mathcal{HL}_\Sigma/\mathcal O_\Sigma)=\varprojlim U_n\]
is pro-unipotent. Its Lie algebra is isomorphic to the primitive elements $\{T\in\mathbb C\langle\langle X\rangle\rangle|\Delta T=T\otimes 1+1\otimes T\}$

\begin{definition}
A realization of hyperlogarithms is a differential $\mathcal O_\Sigma$ algebra $A$ and a nonzero homorphism $\rho:\mathcal{HP}_\Sigma\to A$, since $\ker\rho$ is a differential ideal, $\rho$ is injective, and therefore $\{\rho(w)|w\in X^*\}$ form an $\mathcal O_\Sigma$ basis of the image of $\rho$. The holomorphic realization of hyperlogarithms is
\[\mathcal{HP}_\Sigma\to L_\Sigma,w\mapsto L_w(z)\]
\end{definition}

\begin{corollary}
$\{L_w(z)\}$ form an $\mathcal O_\Sigma$ basis of $L_\Sigma$, the only relations among $L_w(z)$ are the shuffle relations. Each element of $L_\Sigma$ has a primitive in $L_\Sigma$ which is unique up to a constant
\end{corollary}

\begin{definition}
Let $M$ be an $\mathcal O_\Sigma$ module of finite type, equipped with a connection $\nabla$, i.e. a linear map $M\to M$ such that $\nabla(fm)=f'm+f\nabla m$. Under some base $\nabla=\frac{\partial}{\partial z}+P$, where $P$ is a matrix with entries in $\mathcal O_\Sigma$. Define the solutions of $M$ to be the smallest $\mathcal O_\Sigma$ submodule $\Sol(M)$ such that closed under taking primitive, i.e. $\nabla\theta\in \Sol(M)\Rightarrow\theta\in \Sol(M)$. $M$ is free over $\mathcal O_\Sigma$ since $\mathcal O_\Sigma$ is differentially simple
\end{definition}

\begin{lemma}
The following are equivalent
\begin{enumerate}
\item There is a filtration\[0=M_0\subsetneq M_1\subsetneq \cdots\subsetneq M_n=M\]such that $\nabla M_i\subseteq M_{i-1}$ and $(M_i/M_{i-1},\nabla)\cong(\mathcal O_\Sigma,\partial/\partial z)$
\item There is some basis with respect to which that $P$ is strictly upper triangular. We say $M$ is \textit{unipotent}
\item $M$ is spanned by its solutions, i.e. $M=\Sol(M)$
\end{enumerate}
\end{lemma}

\begin{definition}
Suppose $(A,\partial)$ is a finitely generated differential $\mathcal O_\Sigma$ algebra. $A$ is generated by $\mathcal O_\Sigma$ submodule of finite type $M\subseteq A$ with $\partial M\subseteq M$. Then $\partial$ defines a connection on $M$ which restricts to differentiation on $\mathcal O_\Sigma$, further we assume the ring of constants is $\mathbb C$. We say $A$ is unipotent if $M$ is unipotent
\end{definition}

\begin{lemma}
Let $(R,\partial)$ be a differentially simple extension of $\mathcal O_\Sigma$. $R[y]$ is an extension by adjoining an integral, i.e. $\partial y=r\in R$. If $\partial x=r$ has no solution $x\in R$, then $y$ is transcendental over $R$ and $R[y]$ is also differentially simple
\end{lemma}

\begin{proof}

\end{proof}

\begin{theorem}
The following are equivalent
\begin{enumerate}
\item $A$ is unipotent
\item There is an $\mathcal O_\Sigma$ basis of $M$ that can be annihilated by some $D\in\mathcal O_\Sigma[\partial]$ which is a product of $\partial$ and $(z-\sigma_k)$
\item $\Gal(A/\mathcal O_\Sigma)$ acts on $M$ and is unipotent, and $A^{\Gal(A/\mathcal O_\Sigma)}=\mathcal O_\Sigma$
\item There is an embedding $A\xhookrightarrow{i}\mathcal{HL}_\Sigma$
\end{enumerate}
In such case, $\Frac A/\mathbb C(z)$ is an Picard-Vessiot extension, $\Gal(A/\mathcal O_\Sigma)\cong\Gal(\Frac A/\mathbb C(z))$ and $i$ in 4. is unique up to an element in this group
\end{theorem}

\begin{corollary}
$\mathcal{HL}_\Sigma$ is the unipotent closure of $\mathcal O_\Sigma$, i.e. $\displaystyle\mathcal{HL}_\Sigma=\varprojlim_AA$, where $A$ ranges over all finitely generated unipotent differential algebra extensions over $\mathcal O_\Sigma$
\end{corollary}

\begin{definition}
The regularized zeta series at $\sigma_k$ is $Z^{\sigma_k}(X)=L^{\sigma_k}(z)^{-1}L(z)\in\mathbb C\langle\langle X\rangle\rangle$ which is written as
\[Z^{\sigma_k}(z)=\sum_{w\in X^*}\zeta^{\sigma_k}(w)w\]
\end{definition}

\begin{lemma}
\[Z^{\sigma_k}(z)=\lim_{z\to\sigma_k}e^{-x_k\log(z-\sigma_k)}L(z)\]
\end{lemma}


\end{document}