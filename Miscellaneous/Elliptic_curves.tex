\documentclass[main]{subfiles}

\begin{document}

Consider ellipse $\dfrac{x^2}{a^2}+\dfrac{y^2}{b^2}=1$, the circumference is
\[4\int_0^a\sqrt{1+\frac{b^2x^2}{a^2(a^2-x^2)}}dx=4a\int_0^{\frac{\pi}{2}}\sqrt{1-e^2\sin^2\theta}d\theta\]

\begin{definition}
The elliptic integral of the \textit{first} kind is
\[\int_0^\varphi\frac{d\theta}{\sqrt{1-k^2\sin^2\theta}}\]
Let $t=\sin\theta$, $x=\sin\varphi$, we have
\[\int_0^x\frac{dt}{\sqrt{(1-t^2)(1-k^2t^2)}}\]
The elliptic integral of the \textit{second} kind is
\[\int_0^\varphi \sqrt{1-k^2\sin^2\theta}d\theta\]
Let $t=\sin\theta$, $x=\sin\varphi$, we have
\[\int_0^x\frac{\sqrt{1-k^2t^2}}{\sqrt{1-t^2}}dt\]
The elliptic integral of the \textit{third} kind is
\[\int_0^\varphi\frac{d\theta}{(1-n\sin^2\theta)\sqrt{1-k^2\sin^2\theta}}\]
Let $t=\sin\theta$, $x=\sin\varphi$, we have
\[\int_0^x\frac{dt}{(1-nt^2)\sqrt{(1-t^2)(1-k^2t^2)}}\]
These elliptic integrals are called \textit{incomplete}, they are \textit{complete} if $\varphi=\frac{\pi}{2}$
\end{definition}

\begin{theorem}[Legendre's relation]\label{Legendre's relation}
For $k^2+k'^2=1$, $E,E'$ are corresponding complete elliptic integrals of the second kind, $K,K'$ are corresponding complete elliptic integrals of the first kind, then they satisfy the \textit{Legendre's relation}\index{Legendre's relation}
\[KE'+K'E-KK'=\frac{\pi}{2}\]
Equivalently
\[\omega_1\eta_2-\omega_2\eta_1=2\pi i\]
$\omega_1,\omega_2$ are the periods of Weierstrass $\wp$ function, $\eta_1,\eta_2$ are the quasiperiods of Weierstrass zeta function
\end{theorem}

\begin{definition}
An \textit{elliptic integral}\index{Elliptic integral} is of the form
\[\int_c^xR\left(x,\sqrt{P(x)}\right)dx\]
Here $R(x,w)$ is a rational function of $x,w$ and $P(x)$ is a polynomial of degree $3$ or $4$. Every elliptic integral can be reduced into elliptic integrals of the first, second and third kinds
\end{definition}

\begin{definition}
An \textit{abelian integral}\index{Abelian integral} is of the form
\[\int_{z_0}^zR(x,w)dx\]
$R$ is a rational function of $x,w$, and $F(x,w)=0$ for some
\[\varphi_n(x)w^n+\cdots+\varphi_0(x)=0\]
$\varphi_i(x)$ are rational functions of $x$. It is called a \textit{hyperelliptic integral}\index{Hyperelliptic integral} if $F(x,w)=w^2-P(x)$ for some polynomial $P$, note that if degree of $P$ is $3$ or $4$ than it is an elliptic integral
\end{definition}

\begin{definition}
$C$ is a compact algebraic curve of genus $g$, $H^0(X,K)=\mathbb C^g$ is generated by one forms $\omega_1,\cdots,\omega_g$, $K$ is a canonical bundle, the \textit{Abel-Jacobi map}\index{Abel-Jacobi map} is
\begin{align*}
J:C&\to J(C)=\mathbb C^g/\Lambda \\
P&\mapsto \left(\int_{P_0}^P\omega_1,\cdots,\int_{P_0}^P\omega_g\right)\mod\Lambda
\end{align*}
\end{definition}

\begin{theorem}[Abel-Jacobi theorem]
Abel-Jacobi map $J$ is an isomorphism
\end{theorem}

\end{document}