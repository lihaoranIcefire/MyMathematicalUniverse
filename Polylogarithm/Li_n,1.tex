\documentclass[main]{subfiles}

\begin{document}

\[\Li_{1,1}(x,y)=\int_{(0,0)}^{(x,y)}dv_2dv_1+dv_{12}dw_{1,2}\]
Here $w_{1,2}=u_1+v_2-v_1$, then
\begin{align*}
\Li_{n,1}(x,y)&=\int_{(0,0)}^{(x,y)} (dv_2dv_1+dv_{12}dw_{1,2})(du_1)^{n-1} +dv_{12}\left(\sum_{k=1}^{n-1}du_{12}^kdv_2(du_1)^{n-1-k}\right)
\end{align*}
since inductively
\begin{align*}
\Li_{n,1}(x,y)&=\int_{(0,0)}^{(x,y)} \Li_{n-1,1}(x,y)du_1-\Li_n(xy)dv_2 \\
&=\int_{(0,0)}^{(x,y)} \left[(dv_2dv_1+dv_{12}dw_{1,2})(du_1)^{n-2} +dv_{12}\left(\sum_{k=1}^{n-2}du_{12}^kdv_2(du_1)^{n-2-k}\right)\right]du_1 \\
&+dv_{12}(du_{12})^{n-1}dv_2
\end{align*}

Suppose $p$ is the path from $(0,0)$ to $(x,y)$ that give the value of $\Li_{n,1}(x,y)$, $q$ is a loop based at $(x,y)$ that induces monodromy, we can take $p$ to $p(\epsilon)$ to start at $(\epsilon,\epsilon)$ and then take limit $\epsilon\to0$, then $pq$ can be homotopied to some $q(\epsilon)p(\epsilon)$ \\
To compute the monodromy around $x=0$, take $q(\epsilon)$ to be the loop $(x=\epsilon e^{it},y=\epsilon)$
\begin{align*}
\lim_{\epsilon\to0}\int_{q(\epsilon)p(\epsilon)}-\int_{p(\epsilon)}&=0
\end{align*}
To compute the monodromy around $y=0$, take $q(\epsilon)$ to be the loop $(x=\epsilon,y=\epsilon e^{it})$
\begin{align*}
\lim_{\epsilon\to0}\int_{q(\epsilon)p(\epsilon)}-\int_{p(\epsilon)}&=\int_qdv_2\int_pdv_1(du_1)^{n-1}=0
\end{align*}
To compute the monodromy around $x=1$, take $q(\epsilon)$ to be the composition of $(x=(1-t)\epsilon+t(1-\epsilon),y=\epsilon)$, $(x=1-\epsilon e^{it},y=\epsilon)$ and $(x=(1-t)(1-\epsilon)+t\epsilon,y=\epsilon)$
\begin{align*}
\lim_{\epsilon\to0}\int_{q(\epsilon)p(\epsilon)}-\int_{p(\epsilon)}&=0
\end{align*}
To compute the monodromy around $y=1$, take $q(\epsilon)$ to be the composition of $(x=\epsilon,y=(1-t)\epsilon+t(1-\epsilon))$, $(x=\epsilon,y=1-\epsilon e^{it})$ and $(x=\epsilon,y=(1-t)(1-\epsilon)+t\epsilon)$
\begin{align*}
\lim_{\epsilon\to0}\int_{q(\epsilon)p(\epsilon)}-\int_{p(\epsilon)}&=\int_qdv_2\int_pdv_1(du_1)^{n-1}=-2\pi i\Li_n(x)
\end{align*}
To compute the monodromy around $xy=1$, take $q$ to be the loop $(x=x^0,y\text{ such that }\int_qd\log(1-xy)=2\pi i)$
First it's easy to show inductively
\[\int_{(x^0,y^0)}^{(x,y)} dv_2du_2^{n-1}=\Li_n(y)-\sum_{k=1}^n\frac{(\log y-\log y^0)^{n-k}}{(n-k)!}\Li_k(y^0)\]
Thus
\begin{align*}
\lim_{\epsilon\to0}\int_{pq}-\int_{p}&=\sum_{k=0}^{n-1}\int_pdv_{12}(du_{12})^k\int_q(du_{12})^{n-1-k}dv_2
+\int_qdv_{12}du_{12}^{n-1}dv_2 \\
&=(-1)^{n+1}\int_{q^{-1}}dv_2du_{2}^{n-1}dv_{12} \\
&=(-1)^{n+1}\int_{q^{-1}}\frac{\Li_n(y)-\sum_{k=1}^n\frac{(\log y-\log y^0)^{n-k}}{(n-k)!}\Li_k(y^0)}{y-1/x^0}dy \\
&=(-1)^{n}2\pi i\left(-g_n(x^0,y^0)-\Li_n(1/x^0)\right)
\end{align*}
Where
\[g_m(x,y)=\sum_{k=1}^{m}(-1)^{k+1}\frac{\log^{m-k}(xy)}{(m-k)!}\Li_{k}(y)\]
The variation matrix is
\[\Lambda=\left[\begin{smallmatrix}
1\\
\Li_1(y)&1\\
\Li_1(xy)&0&1\\
\Li_{1,1}(x,y)&\Li_1(x)&g_1(x,y)-\Li_1(x^{-1})&1\\
\Li_2(xy)&0&\log x&0&1\\
\Li_{2,1}(x,y)&\Li_2(x)&g_2(x,y)+\Li_2(x^{-1})&\log x&g_1(x,y)&1\\
\vdots&\vdots&\vdots&\vdots&\vdots&\vdots&\ddots\\
\Li_n(xy)&0&\frac{\log^{n-1}(xy)}{(n-1)!}&0&\frac{\log^{n-2}(xy)}{(n-2)!}&0&\cdots&1\\
\Li_{n,1}(x,y)&\Li_n(x)&g_n(x,y)+(-1)^n\Li_n(x^{-1})&\frac{\log^{n-1}x}{(n-1)!}&g_{n-1}(x,y)&\frac{\log^{n-2}x}{(n-2)!}&\cdots&g_1(x,y)&1
\end{smallmatrix}\right]\tau_{n,1}(2\pi i)\]
\[\omega=\begin{bmatrix}
0\\
-dv_2&0\\
-dv_{12}&0&0 \\
&-dv_2&-dw_1&0 \\
&&du_{12}&0&0 \\
&&&du_1&-dv_2&0 \\
&&&&\ddots&\ddots&\ddots \\
&&&&&du_{12}&0&0 \\
&&&&&&du_1&-dv_2&0 \\
\end{bmatrix}\]

\[dg_m(x,y)=g_{m-1}(x,y)du_1-\frac{\log^{m-1}(xy)}{(m-1)!}dv_2\]
Justifies the differential equation $d\Lambda=\omega\Lambda$ \\
Monodromy of $\log^mx/m!$ around $x=0$ is
\[\sum_{k=0}^{m-1}\frac{\log^kx}{k!}\frac{(2\pi i)^{m-k}}{(m-k)!}\]
Monodromy of $\log^m(xy)/m!$ around $x=0$ is
\[\sum_{k=0}^{m-1}\frac{\log^k(xy)}{k!}\frac{(2\pi i)^{m-k}}{(m-k)!}\]
Monodromy of $\Li_n(x^{-1})$ around $x=0$ is $2\pi i\log^{n-1}(x^{-1})/(n-1)!=(-1)^{n-1}2\pi i\log^{n-1}(x)/(n-1)!$ \\
Monodromy of $g_n$ around $x=0$ or $y=0$ is
\begin{align*}
\sum_{k=1}^{m}\sum_{l=1}^{m-k}(-1)^{k+1}\Li_k(y)\frac{\log^{m-k-l}(xy)}{(m-k-l)!}\frac{(2\pi i)^{l}}{l!}&=\sum_{l=1}^{m}\sum_{k=1}^{m-l}(-1)^{k+1}\Li_k(y)\frac{\log^{m-k-l}(xy)}{(m-k-l)!}\frac{(2\pi i)^{l}}{l!} \\
&=\sum_{l=1}^{m-1}\frac{(2\pi i)}{l!}g_{m-l}(x,y)
\end{align*}
Monodromy of $g_n$ around $y=1$ is
\[2\pi i\sum_{k=1}^{m}(-1)^k\frac{\log^{m-k}(xy)}{(m-k)!}\]
The monodromy representation $\rho$ is as follows \\
For monodromy around $x=0$
\[\begin{bmatrix}
1\\
&1\\
&&1\\
&&-1&1\\
&&1&0&1\\
&&0&1&0&1\\
&&\frac{1}{2}&0&1&0&1\\
&&0&\frac{1}{2}&0&1&0&1\\
&&\frac{1}{3!}&0&\frac{1}{2}&0&1&0&1\\
&&\vdots&\vdots&\vdots&\ddots&\ddots&\ddots&\ddots&\ddots\\
&&\frac{1}{(n-1)!}&0&\frac{1}{(n-2)!}&\cdots&\frac{1}{2}&0&1&0&1 \\
&&0&\frac{1}{(n-1)!}&0&\cdots&0&\frac{1}{2}&0&1&0&1
\end{bmatrix}\]
For monodromy around $y=0$
\[\begin{bmatrix}
1\\
&1\\
&&1\\
&&0&1\\
&&1&0&1\\
&&0&0&0&1\\
&&\frac{1}{2}&0&1&0&1\\
&&0&0&0&0&0&1\\
&&\vdots&\vdots&\vdots&\ddots&\ddots&\ddots&\ddots\\
&&\frac{1}{(n-1)!}&0&\frac{1}{(n-2)!}&\cdots&0&1&0&1\\
&&&&&&&&&&1
\end{bmatrix}\]
For monodromy around $x=1$
\[\begin{bmatrix}
1\\
&1\\
&&1&&&\\
&-1&1&1&&\\
&&&&1&\\
&&&&&\ddots\\
&&&&&&1
\end{bmatrix}\]
For monodromy around $y=1$
\[\begin{bmatrix}
1&&&&&&&\\
-1&1&&&&&&\\
&&1&&&&&\\
&&-1&1&&&&\\
&&&\ddots&\ddots\\
&&&&&1&\\
&&&&&-1&1\\
\end{bmatrix}\]
For monodromy around $xy=1$
\[\begin{bmatrix}
1\\
&1\\
-1&&1\\
&&&1\\
&&&&\ddots\\
&&&&&1
\end{bmatrix}\]

\end{document}