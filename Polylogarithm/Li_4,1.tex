\documentclass[main]{subfiles}

\begin{document}

\begin{align*}
\Li_{n,1}(x,y)&=\int (dv_2dv_1+dv_{12}dw_1)(du_1)^{n-1} +dv_{12}\left(\sum_{k=1}^{n-1}du_{12}^kdv_2(du_1)^{n-1-k}\right)
\end{align*}
Here $w_{1}=u_1+v_2-v_1$

To compute the monodromy around $x=0$, take $q(\epsilon)$ to be the loop $(x=\epsilon e^{it},y=\epsilon)$
\begin{align*}
\lim_{\epsilon\to0}\int_{q(\epsilon)p(\epsilon)}-\int_{p(\epsilon)}&=0
\end{align*}
To compute the monodromy around $y=0$, take $q(\epsilon)$ to be the loop $(x=\epsilon,y=\epsilon e^{it})$
\begin{align*}
\lim_{\epsilon\to0}\int_{q(\epsilon)p(\epsilon)}-\int_{p(\epsilon)}&=\int_qdv_2\int_pdv_1\cdots(du_1)^{n-1}=0
\end{align*}
To compute the monodromy around $x=1$, take $q(\epsilon)$ to be the composition of $(x=(1-t)\epsilon+t(1-\epsilon),y=\epsilon)$, $(x=1-\epsilon e^{it},y=\epsilon)$ and $(x=(1-t)(1-\epsilon)+t\epsilon,y=\epsilon)$
\begin{align*}
\lim_{\epsilon\to0}\int_{q(\epsilon)p(\epsilon)}-\int_{p(\epsilon)}&=0
\end{align*}
To compute the monodromy around $y=1$, take $q(\epsilon)$ to be the composition of $(x=\epsilon,y=(1-t)\epsilon+t(1-\epsilon))$, $(x=\epsilon,y=1-\epsilon e^{it})$ and $(x=\epsilon,y=(1-t)(1-\epsilon)+t\epsilon)$
\begin{align*}
\lim_{\epsilon\to0}\int_{q(\epsilon)p(\epsilon)}-\int_{p(\epsilon)}&=\int_qdv_2\int_pdv_1(du_1)^{n-1}=-2\pi i\Li_n(x)
\end{align*}
To compute the monodromy around $xy=1$, take $q$ to be the loop $(x=x^0,y\text{ such that }\int_qd\log(1-xy)=2\pi i)$
\begin{align*}
\lim_{\epsilon\to0}\int_{pq}-\int_{p}&=\sum_{k=0}^{n-1}\int_pdv_{12}(du_{12})^k\int_q(du_{12})^{n-1-k}dv_2
+\int_qdv_{12}du_{12}^{n-1}dv_2 \\
&=(-1)^{n+1}\int_{q^{-1}}dv_2du_{2}^{n-1}dv_{12} \\
&=(-1)^{n+1}\int_{q^{-1}}\frac{\Li_n(y)-\Li_n(y^0)}{y-1/x^0} \\
&=(-1)^n2\pi i(\Li_n(y^0)-\Li_n(1/x^0))
\end{align*}

The variation matrix is
\[\omega=\begin{bmatrix}
0&&&&&&&&&\\
-dv_2&0&&&&&&&&\\
-dv_{12}&&0&&&&&&&\\
&-dv_1&-dw_1&0&&&&&&\\
&&du_{12}&&0&&&&&\\
&&&du_1&-dv_2&0&&&&\\
&&&&du_{12}&&0&&&\\
&&&&&du_1&-dv_2&0&&\\
&&&&&&du_{12}&&0&\\
&&&&&&&du_1&-dv_2&0
\end{bmatrix}\]
\[\Lambda=\left[\begin{smallmatrix}
1\\
\Li_1(y)&1\\
\Li_1(xy)&&1\\
\Li_{1,1}(x,y)&\Li_1(x)&\Li_1(y)-\Li_1(x^{-1})&1\\
\Li_2(xy)&&\log(xy)&&1\\
\Li_{2,1}(x,y)&\Li_2(x)&\log(xy)\Li_1(y)-\Li_2(y)+\Li_2(x^{-1})&\log x&\Li_1(y)&1\\
\Li_3(xy)&&\log^2(xy)/2&&\log(xy)&&1\\
\Li_{3,1}(x,y)&\Li_3(x)&\frac{1}{2}\log^2(xy)\Li_1(y)-\log(xy)\Li_2(y)+\Li_3(y)-\Li_3(x^{-1})&\log^2x/2&\log(xy)\Li_1(y)-\Li_2(y)&\log x&\Li_1(y)&1\\
\Li_4(xy)&&\log^3(xy)/3!&&\log^2(xy)/2&&\log(xy)&&1\\
\Li_{4,1}(x,y)&\Li_4(x)&\frac{1}{3!}\log^3(xy)\Li_1(y)-\frac{1}{2}\log^2(xy)\Li_2(y)+\log(xy)\Li_3(y)-\Li_4(y)+\Li_4(x^{-1})&\log^3x/3!&\frac{1}{2}\log^2(xy)\Li_1(y)-\log(xy)\Li_2(y)+\Li_3(y)&\log^2x/2&\log(xy)\Li_1(y)-\Li_2(y)&\log x&\Li_1(y)&1
\end{smallmatrix}\right]\tau_{1,1}(2\pi i)\]
The monodromy representation $\rho$ is as follows \\
For monodromy around $x=0$
\[\begin{bmatrix}
1&&&&&&&&&\\
&1&&&&&&&&\\
&&1&&&&&&&\\
&&-1&1&&&&&&\\
&&1&&1&&&&&\\
&&&1&&1&&&&\\
&&\frac{1}{2}&&1&&1&&&\\
&&&\frac{1}{2}&&1&&1&&\\
&&\frac{1}{3!}&&\frac{1}{2}&&1&&1&\\
&&&\frac{1}{3!}&&\frac{1}{2}&&1&&1
\end{bmatrix}\]
For monodromy around $y=0$
\[\begin{bmatrix}
1&&&&&&&&&\\
&1&&&&&&&&\\
&&1&&&&&&&\\
&&&1&&&&&&\\
&&1&&1&&&&&\\
&&&&&1&&&&\\
&&\frac{1}{2}&&1&&1&&&\\
&&&&&&&1&&\\
&&\frac{1}{3!}&&\frac{1}{2}&&1&&1&\\
&&&&&&&&&1
\end{bmatrix}\]
For monodromy around $x=1$
\[\begin{bmatrix}
1&&&&&&&&&\\
&1&&&&&&&&\\
&&1&&&&&&&\\
&-1&1&1&&&&&&\\
&&&&1&&&&&\\
&&&&&1&&&&\\
&&&&&&1&&&\\
&&&&&&&1&&\\
&&&&&&&&1&\\
&&&&&&&&&1
\end{bmatrix}\]
For monodromy around $y=1$
\[\begin{bmatrix}
1&&&&&&&&&\\
-1&1&&&&&&&&\\
&&1&&&&&&&\\
&&-1&1&&&&&&\\
&&&&1&&&&&\\
&&&&-1&1&&&&\\
&&&&&&1&&&\\
&&&&&&-1&1&&\\
&&&&&&&&1&\\
&&&&&&&&-1&1
\end{bmatrix}\]
For monodromy around $xy=1$
\[\begin{bmatrix}
1&&&&&&&&&\\
&1&&&&&&&&\\
-1&&1&&&&&&&\\
&&&1&&&&&&\\
&&&&1&&&&&\\
&&&&&1&&&&\\
&&&&&&1&&&\\
&&&&&&&1&&\\
&&&&&&&&1&\\
&&&&&&&&&1
\end{bmatrix}\]

$d\omega_{4,1}(x,y)+\omega_{2}(y)\wedge\omega_{3}(xy)+\omega_{2}(xy)\wedge\omega_3(y)=0$, we have $d\bm{\omega}-\bm{\omega}\wedge\bm{\omega}=0$ for
\[\bm{\omega}=\begin{bmatrix}
0\\
\omega_{2}(xy)&0\\
\omega_{3}(xy)&0&0\\
\omega_{4,1}(x,y)&\omega_3(y)&-\omega_2(y)&0
\end{bmatrix}\]

And the variation matrix should be
\[
\begin{bmatrix}
1\\
\widehat{\mathcal L}_{2}(xy)&1\\
\widehat{\mathcal L}_{3}(xy)&0&1\\
\widehat{\mathcal L}_{4,1}(x,y)&\widehat{\mathcal L}_3(y)&-\widehat{\mathcal L}_2(y)&1\\
\end{bmatrix}
\]

Here
\[\widehat{\mathcal L}_{2}(x)=\Li_{2}(x)+\frac{1}{2}v_1u_1\]
\[\widehat{\mathcal L}_{3}(x)=\Li_{3}(x)-\Li_2(x)u_1-\frac{1}{3}u_1^2v_1\]
\begin{align*}
\widehat{\mathcal L}_{4,1}(x,y)&=\Li_{4,1}(x,y)+\Li_4(xy)v_2-\Li_{3,1}(x,y)u_1+\frac{1}{2}Li_{2,1}(x,y)u_1^2-\frac{1}{6}u_1^3\Li_{1,1}(x,y) \\
&-u_1v_2\Li_3(xy)-\frac{1}{2}u_2v_2\Li_3(xy)+\left(\frac{1}{2} u_1^2 v_2 + \frac{1}{6} u_2^2 v_2 + \frac{1}{2} u_1 u_2 v_2\right)\Li_{2}(xy) \\
&+\frac{1}{30} u_1^4 v_{12} - \frac{1}{30} u_1^3 v_1 v_{12} + \frac{2}{15} u_1^3 v_2 v_{12} + 
  \frac{1}{30} u_2^3 v_{12} v_2 + \frac{1}{30} u_1^3 v_1 v_2 + \frac{1}{5} u_1^2 u_2 v_2 v_{12} + 
  \frac{2}{15} u_1 u_2^2 v_2 v_{12}
\end{align*}

The unipotent monodromy representation would be


\end{document}