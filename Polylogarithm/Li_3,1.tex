\documentclass[main]{subfiles}

\begin{document}

\begin{align*}
\Li_{3,1}(x,y)&=\int\frac{dy}{1-y}\frac{dx}{1-x}\left(\frac{dx}{x}\right)^2+\frac{d(xy)}{1-xy}\left(\frac{dy}{1-y}-\frac{dx}{x(1-x)}\right)\left(\frac{dx}{x}\right)^2 \\
&+\frac{d(xy)}{1-xy}\frac{d(xy)}{xy}\frac{dy}{1-y}\frac{dx}{x}+\frac{d(xy)}{1-xy}\left(\frac{d(xy)}{xy}\right)^2\frac{dy}{1-y} \\
&=\int dv_2dv_1(du_1)^2+dv_{12}dw_1(du_1)^2 +dv_{12}du_{12}dv_2du_1+dv_{12}(du_{12})^2dv_2
\end{align*}
Here we write $w_{1}=u_1+v_2-v_1=\log\frac{x(1-y)}{1-x}$

To compute the monodromy around $x=0$, take $q(\epsilon)$ to be the loop $(x=\epsilon e^{it},y=\epsilon)$, we get $0$. \\
To compute the monodromy around $y=0$, take $q(\epsilon)$ to be the loop $(x=\epsilon,y=\epsilon e^{it})$, we get $0$. \\
To compute the monodromy around $x=1$, take $q(\epsilon)$ to be the composition of $(x=(1-t)\epsilon+t(1-\epsilon),y=\epsilon)$, $(x=1-\epsilon e^{it},y=\epsilon)$ and $(x=(1-t)(1-\epsilon)+t\epsilon,y=\epsilon)$, we get $0$. \\
To compute the monodromy around $y=1$, take $q(\epsilon)$ to be the composition of $(x=\epsilon,y=(1-t)\epsilon+t(1-\epsilon))$, $(x=\epsilon,y=1-\epsilon e^{it})$ and $(x=\epsilon,y=(1-t)(1-\epsilon)+t\epsilon)$, we get $-2\pi i\Li_3(x)$. \\
To compute the monodromy around $xy=1$, take $q$ to be the loop $(x=x^0,y\text{ such that }\int_qd\log(1-xy)=2\pi i)$, we get $-2\pi i\Li_1(\frac{1-xy}{1-x})$

The variation matrix is
\[\Lambda=\left[\begin{smallmatrix}
1\\
\Li_1(y)&1\\
\Li_1(xy)&&1\\
\Li_{1,1}(x,y)&\Li_1(x)&\Li_1(y)-\Li_1(x^{-1})&1\\
\Li_2(xy)&&\log(xy)&&1\\
\Li_{2,1}(x,y)&\Li_2(x)&\log(xy)\Li_1(y)-\Li_2(y)+\Li_2(x^{-1})&\log x&\Li_1(y)&1\\
\Li_3(xy)&&\log^2(xy)/2&&\log(xy)&&1\\
\Li_{3,1}(x,y)&\Li_3(x)&\frac{1}{2}\log^2(xy)\Li_1(y)-\log(xy)\Li_2(y)+\Li_3(y)-\Li_3(x^{-1})&\log^2x/2&\log(xy)\Li_1(y)-\Li_2(y)&\log x&\Li_1(y)&1
\end{smallmatrix}\right]\tau_{3,1}(2\pi i)\]
\[\left[\begin{smallmatrix}
1\\
-I(0;y^{-1};1)&1\\
-I(0;(xy)^{-1};1)&0&1\\
I(0;(xy)^{-1},y^{-1};1)&-I(0;(xy)^{-1};y^{-1})&-I((xy)^{-1};y^{-1};1)&1\\
-I(0;(xy)^{-1},0;1)&0&-I((xy)^{-1};0;y^{-1})&0&1\\
I(0;(xy)^{-1},0,y^{-1};1)&-I(0;(xy)^{-1},0;y^{-1})&-I((xy)^{-1};0,y^{-1};1)&I((xy)^{-1};0;y^{-1})&-I(0;y^{-1};1)&1\\
-I(0;(xy)^{-1},0,0;1)&0&I((xy)^{-1};0,0;1)&0&I((xy)^{-1};0;0)&0&1\\
I(0;(xy)^{-1},0,0,y^{-1};1)&-I(0;(xy)^{-1},0,0;y^{-1})&-I((xy)^{-1};0,0,y^{-1};1)&I((xy)^{-1};0,0;y^{-1})&-I((xy)^{-1};0;0)I(0;y^{-1};1)-I(0;0,y^{-1};1)&I((xy)^{-1};0;0)+I(0;0;y^{-1})&-I(0;y^{-1};1)&1
\end{smallmatrix}\right]\]
\[\omega=\begin{bmatrix}
0&&&&&&&\\
-dv_2&0&&&&&&\\
-dv_{12}&0&0&&&&&\\
&-dv_1&-dw_1&0&&&&\\
&&du_{12}&0&0&&&\\
&&&du_1&-dv_2&0&&\\
&&&&du_{12}&0&0&\\
&&&&&du_1&-dv_2&0
\end{bmatrix}\]
The monodromy representation $\rho$ is as follows \\
For monodromy around $x=0$
\[\begin{bmatrix}
1&&&&&&&\\
&1&&&&&&\\
&&1&&&&&\\
&&-1&1&&&&\\
&&1&&1&&&\\
&&&1&&1&&\\
&&\frac{1}{2}&&&&1&\\
&&&&&1&&1
\end{bmatrix}\]
For monodromy around $y=0$
\[\begin{bmatrix}
1&&&&&&&\\
&1&&&&&&\\
&&1&&&&&\\
&&&1&&&&\\
&&&&1&&&\\
&&&&&1&&\\
&&&&&&1&\\
&&&&&&&1
\end{bmatrix}\]
For monodromy around $x=1$
\[\begin{bmatrix}
1&&&&&&&\\
&1&&&&&&\\
&&1&&&&&\\
&&&1&&&&\\
&&&&1&&&\\
&&&&&1&&\\
&&&&&&1&\\
&&&&&&&1
\end{bmatrix}\]
For monodromy around $y=1$
\[\begin{bmatrix}
1&&&&&&&\\
&1&&&&&&\\
&&1&&&&&\\
&&&1&&&&\\
&&&&1&&&\\
&&&&&1&&\\
&&&&&&1&\\
&&&&&&&1
\end{bmatrix}\]
For monodromy around $xy=1$
\[\begin{bmatrix}
1&&&&&&&\\
&1&&&&&&\\
&&1&&&&&\\
&&&1&&&&\\
&&&&1&&&\\
&&&&&1&&\\
&&&&&&1&\\
&&&&&&&1
\end{bmatrix}\]

$dw_{3,1}-w_2(y)\wedge w_2(xy)=0$. So $\xi=w_{3,1}-\frac{1}{2}(L_2(y)\wedge w_2(xy)-w_2(y)\wedge L_2(xy))$ should be closed. So we want $\widehat L_{3,1}$ such that $\widehat dL_{3,1}=\xi$

$d\omega_{3,1}(x,y)+\omega_2(y)\wedge\omega_2(xy)=0$, we have $d\bm{\omega}-\bm{\omega}\wedge\bm{\omega}=0$ for
\[\bm{\omega}=\begin{bmatrix}
0\\
\omega_2(xy)&0\\
\omega_{3,1}(x,y)&-\omega_2(y)&0
\end{bmatrix}\]

And the variation matrix should be
\[
\begin{bmatrix}
1\\
\widehat{\mathcal L}_2(xy)&1\\
\widehat{\mathcal L}_{3,1}(x,y)&-\widehat{\mathcal L}_2(y)&1\\
\end{bmatrix}
\]

Here
\[\widehat{\mathcal L}_2(x)=\Li_2(x)-\frac{1}{2}\Li_1(x)\log x\]
\begin{align*}
\widehat{\mathcal L}_{3,1}(x,y)&=\Li_{3,1}(x,y)+\Li_3(xy)v_2-\Li_{2,1}(x,y)u_1+\frac{1}{2}\Li_{1,1}(x,y)u_1^2-\frac{1}{2}\Li_2(xy)v_2(2u_1+v_1)\\
&+\frac{1}{4}(-\frac{1}{2}v_1u_1^2v_2+\frac{1}{2}v_1u_1^2v_{12}-\frac{1}{2}u_1^3v_{12}-\frac{3}{2}u_1^2v_2v_{12}-\frac{3}{2}u_1v_2u_2v_{12}-\frac{1}{2}v_2u_2^2v_{12})
\end{align*}

The unipotent monodromy representation would be

Monodromy along the commutator of $x=0,xy=1$
\[
\begin{bmatrix}
1\\
-1&1\\
\frac{1}{2}&&1
\end{bmatrix}
\]

Monodromy along the commutator of $y=0,y=1$
\[
\begin{bmatrix}
1\\
&1\\
&-1&1
\end{bmatrix}
\]

Monodromy along the commutator of $y=0,xy=1$
\[
\begin{bmatrix}
1\\
-1&1\\
&&1
\end{bmatrix}
\]

Monodromy along all other commutators
\[
\begin{bmatrix}
1\\
&1\\
&&1
\end{bmatrix}
\]

\end{document}